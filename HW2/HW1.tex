\documentclass{homework}

\usepackage[table]{xcolor}
\usepackage{booktabs}
\newcommand{\myname}{Kristof Nemeth}
\newcommand{\assignment}{Homework 2}
\newcommand{\duedate}{}
\newcommand{\course}{Coding Theory}

\renewcommand{\C}{\mathcal{C}}
\setcounter{MaxMatrixCols}{29}

\begin{document}

\begin{problem}
  Consider the $[15,7,5]$ 2-error correcting BCH code, whose parity check matrix
  is expressed over $GF(2^4)$ by:
  \[
    H = \begin{pmatrix}
      1 & \alpha & \alpha^2 & \cdots & \alpha^i & \cdots & \alpha^{14}\\
      1 & \alpha^3 & \alpha^6 & \cdots & \alpha^{3i} & \cdots & \alpha^{12}
    \end{pmatrix}
  \]
  \begin{itemize}
    \item
      Write the columns of H in binry form

      \[
        H = \left(\begin{array}{ccccccccccccccc}
            0&0&0&1&0&0&1&1&0&1&0&1&1&1&1\\
            0&0&1&0&0&1&1&0&1&0&1&1&1&1&0\\
            0&1&0&0&1&1&0&1&0&1&1&1&1&0&0\\
            1&0&0&0&1&0&0&1&1&0&1&0&1&1&1\\
            \midrule
            0&1&1&1&1&0&1&1&1&1&0&1&1&1&1\\
            0&0&1&0&1&0&0&1&0&1&0&0&1&0&1\\
            0&0&0&1&1&0&0&0&1&1&0&0&0&1&1\\
            1&0&0&0&1&1&0&0&0&1&1&0&0&0&1\\
        \end{array}\right)
      \]
    \item
      A message $y$ is received, the syndrome is computed and is found to be $S
      = (0,1,0,0,0,1,1,1)^{tr}$. What can you say about any error(s) that have
      occured during transmission.

      $z_1 = (0100) = \alpha^2$ and $z_2 = (0111) = \alpha^{10} \ne z_1^3$. If
      we solve the equation  \[
        x^2 + \alpha^2x+(\frac{\alpha^{10}}{\alpha^2}+(\alpha^2)^2) = x^2 +
        \alpha^2x+\alpha^{13} = 0
      \]
      and find that the quadratic has no roots. Therefore, at least three errors
      have occured
    \item
      Repeat with syndrome $S = (1,0,1,1,1,1,0,0)^{tr}$

      $z_1 = (1011) = \alpha^7$ and $z_2 = (1100) = \alpha^{6} = z_1^3$.
      And so there is a single error at $i = z_1 = 7$
    \item
      Repeat with syndrome $S = (0,1,1,1,0,1,1,0)^{tr}$

      $z_1 = (0111) = \alpha^{10}$ and $z_2 = (0110) = \alpha^{5} \ne z_1^3$.
      If
      we solve the equation  \[
        x^2 + \alpha^{10}x+(\frac{\alpha^5}{\alpha^{10}}+(\alpha^{10})^2) = x^2 +
        \alpha^{10}x+1 = 0
      \]
      The equation has two roots at $i = 3,12$, and so there are two errors at
      those locatons.

  \end{itemize}
\end{problem}

\begin{problem}

\end{problem}

\begin{problem}
  Suppose that $\F$ is a field of order $p^m$, $\alpha$ a primitive element of
  $\F$ and suppose that $C_s$ is the cyclotomic coset of $s$ modulo $p^m-1$
  \begin{itemize}
    \item Prove that \[
        \prod\limits_{j\in C_s} (x-a^j) \in \Z_p[x]
      \]

      Let $C_s$ be the cyclotomic coset of $s$, and let $M^{(s)}(x)$ be the minimal polynmial of $a^s$. Then
      for any $n$, $sp^n \in C_s$. Additionally, since $\F$ has order $p^m$, it has
      characteristic $p$. That is, we can easily show that $(a+b)^p = a^p+b^p$ using
      the Binomial theorem. Hence, for any polynomial $f \in \F$, $f(\beta^p) =
      f(\beta)^p$. So
      \[
        M^{(s)}(a^{sp^n})=M^{(s)}((a^s)^{p^n})=M^{(s)}(a^s)^{p^n}=0
      \]
      So we have shown that for any $i,j\in C_s$, $M^{(i)}(x)=M^{(j)}(x)$. Since
      each $j \in C_s$ satisfies $M^{(s)}(a^j) = 0$, we have  \[
        \prod\limits_{j\in C_s} (x-a^j) \text{  divides  } M^{(s)}(x)
      \] 

    \item
      Illustrate the above result with $p^m = 16$, and $s=3$.

      We will use $\F = GF(16)$. We have $C_s = \{3,6,12,9\}$. Then 
      \[
        \prod\limits_{j\in C_s} (x-a^j) = (x-a^3)(x-a^6)(x-a^{12})(x-a^9) =
        x^4+x^3+x^2+x+1 \in Z_2[x]  
      \] 
      (the above was computed using Julia's Nemo Library)

  \end{itemize}
\end{problem}

\begin{problem}
  Prove that
  \begin{itemize}
    \item $A_q(n,1) = q^n$
      Consider all the q-ary words of length n. There are exactly $q^n$ of them.
      Additionally, distinct words always have a distance of at least one. Hence
      the code $C$ containing all the q-ary words of length n is a $(n,q^n,q)$
      code. Since there are no more words to add, we have $A_q(n,1) = q^n$.

    \item
      Assume we have a $(n,M,n)$ code such that $M > q$. Then, by the pigeonhole
      principle, there must be two words, $s_1,s_2$ that have the same first
      digit. Then $d(s_1,s_2) \le n-1$, and so we have a contradiction. Hence
      $A_q(n,n) \le q$. Additionally, the repetiton code of length  $n$ is a
      $(n,q,n)$ code, and so $A_q(n,n) = q$.

  \end{itemize}
\end{problem}

\begin{problem}
  Construct, if possible binary $(n,M,d)$ codes with the following parameters:

  \begin{itemize}
    \item (6,2,6)

      As discussed in the previous problem, this is the binary repetition code
      of length 6. That is , the two code words are $(000000)$, and $(111111)$.
    \item (3,8,1)

      As discussed in the previous problem, this is the code containing all
      binary words of length 3.
    \item (4,8,2)

      Consider the code $C =
      \{(0000),(1111),(1100),(0011),(1010),(0101),(1001),(0110)\}$

    \item (5,3,4)

      These parameters fail the Plotkin bound, since $3 > 2\floor{\frac{4}{3}} =
      2$
    \item (8,30,3)

  \end{itemize}
\end{problem}

\begin{problem}
  Let $C$ be a binary $(n,M,d)$ code. Now consider that there are $N$
  words in $C$ that have a 1 in the first column. Then there are $M-N$ words
  with 0, and one of either $N$ or $M-N$ is $\ge M/2$. Now, let us delete only
  the words with a 1, or 0 in the first column (whichever one is fewer). We are
  left with a $(n,M',d)$ code, where $M' \ge M/2$. Now we can simply puncture
  the first column, and the distance between the words will not change, since
  all the words remaining agree in the first column.

  Now let $M = A_2(n,d)$. Then there exists  $M' \ge M/2$. Hence  $A_2(n,d) = M
  \le 2M' \le 2A_2(n-1,d)$.

\end{problem}

\end{document}
